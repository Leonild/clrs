\documentclass{article}
\usepackage{amsmath,clrscode}
\usepackage{booktabs}
\title{Iterative vs. Memoized \proc{LCS-Length}}
\author{Peter Danenberg}
\begin{document}
\maketitle

Iterative \proc{LCS-Length} (figure \ref{fig:iter}), whose complexity
is asymptotically equivalent, performs more calculations than memoized
\proc{LCS-Length} (figure \ref{fig:memo}).
\begin{figure}[ht]
  \[
    \begin{array}{c|ccccccccc}
      \toprule i, j
      & 0 & 1 & 0 & 1 & 1 & 0 & 1 & 1 & 0\\
      \hline
      1 & 0 & 1 & 1 & 1 & 1 & 1 & 1 & 1 & 1\\
      0 & 1 & 1 & 2 & 2 & 2 & 2 & 2 & 2 & 2\\
      0 & 1 & 1 & 2 & 2 & 2 & 3 & 3 & 3 & 3\\
      1 & 1 & 2 & 2 & 3 & 3 & 3 & 4 & 4 & 4\\
      0 & 1 & 2 & 3 & 3 & 3 & 4 & 4 & 4 & 5\\
      1 & 1 & 2 & 3 & 4 & 4 & 4 & 5 & 5 & 5\\
      0 & 1 & 2 & 3 & 4 & 4 & 5 & 5 & 5 & 6\\
      1 & 1 & 2 & 3 & 4 & 5 & 5 & 6 & 6 & 6\\
      \bottomrule
    \end{array}
    \]
  \caption{Lengths calculated by iterative \proc{LCS-Length}}
  \label{fig:iter}
\end{figure}
\begin{figure}[ht]
  \[
    \begin{array}{c|cccccccccc}
      \toprule i, j
      & 0 & 1 & 0 & 1 & 1 & 0 & 1 & 1 & 0\\
      \hline
      1 & \, & 1 & \, & 1 & 1 & \, & \, & \, & \,\\
      0 & 1 & 1 & 2 & 2 & 2 & \, & \, & \, & \,\\
      0 & 1 & \, & 2 & 2 & \, & 3 & \, & \, & \,\\
      1 & \, & 2 & \, & 3 & 3 & \, & 4 & \, & \,\\
      0 & \, & \, & 3 & 3 & \, & 4 & 4 & \, & \,\\
      1 & \, & \, & \, & \, & 4 & \, & 5 & 5 & \,\\
      0 & \, & \, & \, & \, & \, & 5 & 5 & \, & 6\\
      1 & \, & \, & \, & \, & \, & \, & \, & 6 & 6\\
      \bottomrule
    \end{array}
    \]
  \caption{Lengths calculated by memoized \proc{LCS-Length}}
  \label{fig:memo}
\end{figure}
\end{document}
